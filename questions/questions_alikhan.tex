\documentclass[12pt, a4paper]{article}
\usepackage{amsfonts, amsmath, amsthm, amssymb}
\usepackage[left=1cm,right=1cm,top=1cm,bottom=2cm]{geometry}
\usepackage{paralist}
\usepackage[russian]{babel}
\usepackage[utf8]{inputenc}
\usepackage{graphicx}
\usepackage{verbatim}

\newcounter{problem}
\newcommand{\problem}{\par \bigskip \refstepcounter{problem}%
\textbf{Задача №\arabic{problem}.} }

\def \Problem#1{\par \bigskip \textbf{Задача №{#1}. }}
\def \solution{\par \bigskip \textbf{Решение. }}
\def \solutionI{\par \bigskip \textbf{Первое решение. }}
\def \solutionII{\par \bigskip \textbf{Второе решение. }}
\def \solutionIII{\par \bigskip \textbf{Третье решение. }}
\def \Lemma{\par \bigskip \textbf{Лемма. }}
\def \lemma#1{\par \bigskip \textbf{Лемма {#1}. }}
\def \proof{\par \bigskip \textbf{Доказательство. }}
\def \answer{\par \bigskip \textbf{Ответ. }}

\DeclareRobustCommand{\divby}{%
  \mathrel{\text{\vbox{\baselineskip.65ex\lineskiplimit0pt\hbox{.}\hbox{.}\hbox{.}}}}%
}

\newcommand{\N}{\mathbb{N}}
\newcommand{\Z}{\mathbb{Z}}
\newcommand{\R}{\mathbb{R}}
\newcommand{\Comp}{\mathbb{C}}
\newcommand{\Q}{\mathbb{Q}}
\newcommand{\B}{\mathfrak{B}}
\newcommand{\F}{\mathbb{F}}
\newcommand{\E}{\mathbb{E}}
\newcommand{\D}{\mathbb{D}}
\newcommand{\norm}[1]{\left\lVert#1\right\rVert}

\DeclareMathOperator{\rank}{rank \ }
\DeclareMathOperator{\range}{range \ }
\DeclareMathOperator{\nulll}{null \ }
\DeclareMathOperator{\nullity}{nullity \ }
\DeclareMathOperator{\Tr}{Tr \ }
\DeclareMathOperator{\adj}{adj}
\DeclareMathOperator{\spann}{span}
\DeclareRobustCommand{\ker}{ker}
\DeclareRobustCommand{\modulo}{\ mod \ }

\usepackage{amsmath}
\usepackage{systeme}

\usepackage{amsmath}
\DeclareMathOperator{\sign}{sign}

\usepackage{listings}

\usepackage[utf8]{inputenc}

\usepackage{xcolor}

\definecolor{codegreen}{rgb}{0,0.6,0}
\definecolor{codegray}{rgb}{0.5,0.5,0.5}
\definecolor{codepurple}{rgb}{0.58,0,0.82}
\definecolor{backcolour}{rgb}{0.95,0.95,0.92}

\lstdefinestyle{mystyle}{
    backgroundcolor=\color{backcolour},   
    commentstyle=\color{codegreen},
    keywordstyle=\color{blue},
    numberstyle=\tiny\color{codegray},
    stringstyle=\color{codepurple},
    basicstyle=\ttfamily\footnotesize,
    breakatwhitespace=false,         
    breaklines=true,                 
    captionpos=b,                    
    keepspaces=true,                 
    numbers=left,                    
    numbersep=5pt,          
    showspaces=false,
    showstringspaces=false,
    showtabs=false,
    tabsize=2
}

\lstset{style=mystyle}

%\begin{lstlisting}[language = C++]
%\end{lstlisting}

\title{Вопросы по докладу}
\author{Зиманов Алихан}
\date{\today}
%\maketitle
%\pagenumbering{gobble}
\setcounter{problem}{0}

\begin{document}

\maketitle

\par \bigskip \textbf{Вопрос. } Что такое fine-tuning? Что такое parameter-efficient fine-tuning и зачем он нужен?

\par \bigskip \textbf{Ответ. } Fine-tuning --- использование уже обученной, но на другую задачу модели как инициализацию весов для новой модели, решающей текущую задачу. Parameter-efficient fine-tuning --- fine-tuning, который обучает лишь маленькую часть весов модели, тем самым позволяя делить большую часть исходной модели между разными задачами и меняя между ними только эту малую часть обучаемых весов.

\par \bigskip \textbf{Вопрос. } Что такое continual learning и почему его не стоит использовать для решения нескольких задач в случайном порядке?

\par \bigskip \textbf{Ответ. } Continual learning это подход обучения одной модели для решения нескольких задач. В нём одной и той же модели последовательно даются задачи и мы дообучаем её решать текущую задачу. Этот плохо подходит для решения задач в случайном порядке потому что когда мы обучаем нашу модель на новую задачу, её способность решать предыдущую утрачивается, что приводит к заметному ухудшению качества модели.

\par \bigskip \textbf{Вопрос. } В чём основная идея adapter tuning и как его применить к трансформеру?

\par \bigskip \textbf{Ответ. } Adapter tuning заключается в том, что мы добавляем adapter слои между слоями исходной модели и далее обучаем только веса этих добавленных слоев, а веса исходной модели не меняем. Adapter слой состоит из трех преобразований: проекция входного вектора в меньшее пространство, нелинейность, проекция полученного вектора обратно в пространство исходной размерности. Для применения adapter tuning в трансформере надо вставить adapter слои после каждого feed-forward слоя.

\par \bigskip \textbf{Вопрос. } В чём заключается идея prefix-tuning? Как стоит обучать префиксы-контексты?

\par \bigskip \textbf{Ответ. } В моделях NLP часто используются векторы контекста, которые призваны подсказывать модели следующие токены. Prefix-tuning утилизирует эту идею и с помощью специального префикса входных данных подсказывает модели какую именно задачу она сейчас должна решать. Если обучать веса префиксов $P_{\theta}$ явно в лоб, то получится слабенькое качество, поэтому можно заменить их скрытыми префиксами $P_{\theta}'$ и небольшой нейронной сетью $MLP_{\theta}$ так, что $P_{\theta}[i, :] = MLP_{\theta}(P_{\theta}'[i, :])$ и теперь обучать $P_{\theta}'$ и веса у $MLP_{\theta}$.

\par \bigskip \textbf{Вопрос. } Выпишете функцию, которую мы хотим максимизировать во время fine-tuning и объясните все обозначения (по-хорошему, формулировку из метода LoRA).

\par \bigskip \textbf{Ответ. } Функция выглядит как: \[ \max_{\Delta \Phi} \sum_{(x, y) \in Z} \sum_{t = 1}^{|y|} \log \left( P_{\Phi_0 + \Delta \Phi} (y_t | x, y_{< t}) \right) . \]

\begin{itemize}
    \item $\Phi_0$ -- веса исходной модели;
    \item $\Delta \Phi$ -- изменение весов исходной модели во время fine-tuning;
    \item $Z$ -- датасет новой задачи и $(x, y) \in Z$ -- пара входных и выходных данных задачи;
    \item $P_{\Phi_0 + \Delta \Phi}$ -- модель после обучения на новую задачу и $P_{\Phi_0 + \Delta \Phi} (y_t | x, y_{< t})$ -- соответствующая вероятность получить токен $y_t$ при условии токенов из $x$ и $y_{< t}$.
\end{itemize}

\par \bigskip \textbf{Вопрос. } В чём заключается метод LoRA для одной матрицы? К каким матрицам трансформера стоит применять этот метод?

\par \bigskip \textbf{Ответ. } Допустим у нас есть матрица $W_0$ из исходной модели. Добавим в модель рядом с $W_0$ две низкоранговые матрицы $A$ и $B$ так, чтобы при входе $x$ данная часть модели выдавала $W_0 x + B A x$ и будем обучать только эти добавленные матрицы $A$ и $B$. Матрицы $A$ и $B$ играют роль низкорангового приближения матрицы $\Delta W$, которая является изменением матрицы $W_0$ во время честного fine-tuning. В трансформере эти низкоранговые приближения стоит применять к матрицам слоев self-attention.

\end{document}
